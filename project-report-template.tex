%
% File naacl2019.tex
%
%% Based on the style files for ACL 2018 and NAACL 2018, which were
%% Based on the style files for ACL-2015, with some improvements
%%  taken from the NAACL-2016 style
%% Based on the style files for ACL-2014, which were, in turn,
%% based on ACL-2013, ACL-2012, ACL-2011, ACL-2010, ACL-IJCNLP-2009,
%% EACL-2009, IJCNLP-2008...
%% Based on the style files for EACL 2006 by 
%%e.agirre@ehu.es or Sergi.Balari@uab.es
%% and that of ACL 08 by Joakim Nivre and Noah Smith

\documentclass[11pt,a4paper]{article}
%\usepackage[hyperref]{naaclhlt2019}
\usepackage{times}
\usepackage{latexsym}

\usepackage{url}

%\aclfinalcopy % Uncomment this line for the final submission
%\def\aclpaperid{***} %  Enter the acl Paper ID here

%\setlength\titlebox{5cm}
% You can expand the titlebox if you need extra space
% to show all the authors. Please do not make the titlebox
% smaller than 5cm (the original size); we will check this
% in the camera-ready version and ask you to change it back.

\newcommand\BibTeX{B{\sc ib}\TeX}

\begin{document}
\title{Your Project Title and Author Names}
%\maketitle
%\begin{abstract}

%A one paragraph description of your project including the key outcomes.

%\end{abstract}

\maketitle
\section{Project Overview (10 points)}

\paragraph{}Paragraph 1: The first paragraph is a funnel. You start with the broadest scope and quickly narrow down to the specific version of the problem you are tackling. 
First state the problem area you are in working in the broadest terms and motivate it. That is state why this problem area is important. Usually you can use applications or other impact of solutions for this problem area.  Then state why this isnt an easy thing to do. What is the basic challenge in this problem space? In other words you can state why this problem area hasnt been solved. 
You can then make a statement about the scope you are considering and state your overall goal within this problem space. You should very clearly state  the overall goal of your project and describe the task you are trying to solve in terms of inputs and outputs.  


\paragraph{}Paragraph 2: I can this the gap paragraph. What has been done about this type of problem before? There should be some related work. Sometimes there may not be work that solves your exact problem but that is ok. You should find the closest set of solutions and describe them at a high-level here.  What type of approaches has been attempted for this problem? Then, provide a line or two describing why the existing solutions are inadequate. 
End this paragraph with a line that states at a high-level what needs to be done. 

\paragraph{}Paragraph 3: Give your proposed ideas here. What are the ideas/methods you are trying to address this problem? Now you have stated the gap(s) in the previous paragraph.
Your goal here should to be give a high-level description or intuition for what ideas are likely to work in addressing the gap and why. 

\paragraph{}Paragraph 4-6: Detailing the ideas. Use this and one or two subsequent paragraphs to explain the ideas in a some more detail. For each idea you will specify the steps involved (or components that needed to be developed).
If you are using existing implementations or previous papers for the ideas, cite them here. Make sure that you tie back to how each idea addresses the gap in terms of the specifics. 

\paragraph{}Paragraph 7: State how you will evaluate the ideas. Describe the main evaluation questions that need to be addressed to demonstrate/assess the usefulness of your ideas. 
State what systems you will use to implement, datasets you will use to evaluate, and what evaluation measures you will be using. Identify what kind of analyses you will be doing in addition to the main 
evaluation measures which will help you understand the utility of your ideas under different settings. 

\paragraph{}Paragraph 8: Summarize your main findings here. What is the high-level finding from your experiments.


\section{Ideas (10 points)}
Describe in your own words the ideas you tried for the task. Please describe your ideas and how you implement them in detail. Take a page if you need for each for describing each idea. Use figures if it helps to describe your idea. See  

\subsection{Idea 1}

Describe idea 1 here. 

\subsection{Idea 2}

Describe idea 2 here.

\subsection{Idea 3}

Describe idea 3 here.

\section{Experimental Setup (10 points)}

In this section you will describe what models you used, the datasets used, and list the evaluation metrics. 

\subsection{Models}
Describe each model you used at a high-level. What type of model is it? An RNN? A transformer? How many layers do you have in your model? Was it pre-trained? Or trained from scratch. 

If you trained any models, describe the training settings. How many epochs did you train the model for, how long . List the main hyper-parameters (e.g. learning rate, batch size, optimizer used etc.). 

\subsection{Dataset} What is the dataset you are using? What is the input/output format for each instance in your dataset. How many training instances are there? How many test instances did you use?

\subsection{Evaluation Metrics} Say how you evaluated your model. Human evaluation or automatic evaluation? What are the metrics you use to compare the different models. Explain each metric. 

If you use human evaluation for the output, describe what criteria you used to evaluate the outputs. If you use manual evaluation you only need to do this on a small number of examples (say 25 examples for each method).

\section{Results (40 points for implementation and results)}

In this section describe the main results you obtained using tables or plots. Show the best result you were able to obtain under each idea.

For example, if you used fine-tuning DistilBERT, Zero Shot GPT-3, and Few Shot GPT-3 as the main ideas for the PerSent task, your table should look something like Table~\ref{tab:results}

\begin{table}[t!]
\begin{tabular}{l|l|l|l}
\hline
Method & Precision & Recall & F1\\
\hline
DistilBERT & 0.40 & 0.30 & 0.34 \\
GPT-3(ZS) & 0.50 & 0.60 & 54 \\
GPT-3(FS) & \textbf{0.55} & \textbf{0.65} & \textbf{0.59}\\
\hline
\end{tabular}
\caption{Comparison of the three methods on the Fixed test set of the PerSent dataset: GPT-3 in the few-shot achieves the best performance across all measures. GPT-3(ZS) is the zero shot use of GPT-3 and GPT-3(FS) is the few-shot method with K=3 examples.}
\label{tab:results}
\end{table} 

\begin{enumerate}

\item You can report results on training, validation (development) and the test sets in separate tables. Note that the main result you should explain is the one for the test set. 

\item If you tried different hyper-parameter settings, different prompt methods, or any variations for each method show results for those in separate tables.
\end{enumerate}


\section{Analysis and Discussion (25 points)}

Take one model from the ideas you tried above and do a manual analysis of the outputs of this model. Your goal here is to try and understand what kinds of inputs this model succeeds and fails on. You are free to explore how to do this analysis. This section should include the following:
\begin{enumerate}
\item Identify some types of failures of the model.
\item Come up with three hypotheses for the kinds of inputs the model fails or succeeds on.
\item For each hypothesis include an example input and the model's output. Use the example to make your point.
\item If your hypothesis is true then try modifying the input to show that the model actually behaves in the way you anticipated. It is ok if it doesn't. This means your hypothesis isn't correct or it is not easy to test your hypothesis in this way.
\end{enumerate}

\section{Code}

In this section, please provide a {\bf google drive link to your packaged code} or give us a link to your github repo hosting these.  

The code itself should be structured with a README that clearly specifies the following:
\begin{enumerate}
\item List the original source for your code base. Include the URL to the original source. 
\item The list of files that you modified and the specific functions within each file you modified for your project. 
\item A list of commands that provide how you train and test your baseline and the systems you built.
\item If you trained any models please include the trained models and links to the data you trained the models on.
\item If you used prompts please put the prompts in a text file and include them.
\item A list of the major software requirements that are needed to run your system. (E.g. Tensorflow 2.3, Python 243.12, CUDA abd2.0, nltk-2401.11, allen-nlp 5.0). 
\end{enumerate}
These descriptions should be adequate enough to help anyone who wants to run your system. 

\textbf{If we need additional information, we will ask you for it when grading.}

Please include the above information in this report in a succinct fashion (the actual README can be detailed).


\section{Learning Outcomes}
List what you learnt from doing this project. 

\section{Contributions}
List what each member contributed to this project. Note that the entire team will get the same grade.
\end{document}